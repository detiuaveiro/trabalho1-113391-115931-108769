\documentclass{article}
\usepackage[T1]{fontenc} % Fontes T1
\usepackage[utf8]{inputenc} % Input UTF8
\usepackage[backend=biber, style=ieee]{biblatex} % para usar bibliografia
\usepackage{csquotes}
\usepackage[portuguese]{babel} %Usar língua portuguesa
\usepackage{blindtext} % Gerar texto automaticamente
\usepackage[printonlyused]{acronym}
\usepackage{hyperref} % para autoref
\usepackage{indentfirst}
\usepackage{fancyhdr}
\usepackage{hyperref}


\usepackage{graphicx} % Required for inserting images
\usepackage{float} % For precise figure placement using [H]

\title{TAD Image8bit - Algoritmo e Estruturas de Dados}
\author{
  Tiago Soure: 108769 \\
  David Pelicano: 113391 \\
  Joaquim Martins: 115931 \\ 
}
\date{November 2023}

\usepackage{amsmath}
\begin{document}

\maketitle

\begin{figure}[H]
  \centering
  \includegraphics[scale=0.5]{image.png}
\end{figure}  

\newpage
\renewcommand\contentsname{Índice}
\tableofcontents 
\newpage

\pagenumbering{arabic}

\section{Aspetos Gerais}
Neste primeiro projeto de Algoritmos e Estruturas de Dados do primeiro semestre de 2023/2024 foca no desenvolvimento e análise do TAD image8bit. Este TAD é crucial para a manipulação eficiente de imagens de níveis de cinza, apresentando desafios específicos, como a procura por subimagens e a aplicação de filtros de desfoque.

\section{Apresentação}
\subsection{ImageLocateSubImage()}

A função ImageLocateSubImage(Image img1, int* px, int* py, Image img2) procura uma sub-imagem(img2) dentro
de uma imagem maior(img1). Esta percorre todas as posições possíveis na imagem img1 onde a subimagem img2
poderia se encaixar. 
Para cada posição, ela chama a função ImageMatchSubImage() para verificar se a subimagem img2 
corresponde à parte da imagem img1 na posição atual.
Se uma correspondência é encontrada, a função define os valores dos inteiros apontados por px e
py para as coordenadas x e y da posição correspondente na imagem img1 e a função ImageLocateSubImage() retorna 1.
Se a função percorre todas as posições possíveis na imagem img1 sem encontrar uma correspondência,
ela retorna 0, indicando que a subimagem img2 não foi encontrada na imagem img1. Nesse caso, os valores
dos inteiros apontados por px e py não são alterados.
Nos testes, para retornar 1, foi usada a função ImagePaste(Image img1, int x, int y, Image img2),
para colar uma imagem(img2) numa posição específica(x,y) dentro de uma imagem maior(img1), copiando
cada pixel de img2 para a posição correspondente na img1.

\subsection{ImageBlur()}

A função ImageBlur() será usada para aplicar um efeito de desfoque numa imagem.
Esta percorre cada pixel da imagem e calcula a média dos pixeis numa janela à volta do pixel atual.
Contudo, foram feitas dois algoritmos, um mais básico e outro melhorado, ou seja foram criadas 2 funções:
ImageBlur() e ImageBlurv2(). A principal diferença entre as duas funções é a maneira como estas calculam essa média.

Para o algoritmo mais básico, ou seja, para a função ImageBlur(), é usado um método direto para calcular a média.
A função percorre cada pixel na área ao redor do píxel atual, soma os seus valores e divide pelo número total de pixeis na área. 
Este processo é realizado para cada píxel na imagem. A função utiliza uma imagem temporária para
 armazenar os novos valores dos pixeis, de modo a não modificar os valores dos píxeis que ainda 
 precisam ser processados.


Para a função ImageBlurv2() foi utilizada uma técnica denominada imagem integral para calcular a média de 
 forma mais eficiente. Uma imagem integral é uma matriz na qual o valor em cada ponto é a soma de 
 todos os pixeis acima e à esquerda do ponto atual, incluindo o próprio ponto. Isto possibilita o 
 cálculo da soma de qualquer retângulo na imagem em tempo constante, apenas subtraindo e adicionando 
 os valores apropriados da imagem integral. Este método torna o cálculo da média significativamente mais
  rápido, especialmente para imagens grandes.












\section{Análise da complexidade}
\subsection{ImageLocateSubImage()}





Após concluir o desenvolvimento do TAD, é necessário analisar a eficiência computacional do
algoritmo desenvolvido para a função ImageLocateSubImage() e da função ImageBlur()/ImageBlurv2().

\subsubsection{Dados experimentais}

Para obter os dados experimentais, realizou-se uma sequência de testes, com imagens de diferentes tamanhos, onde registou-se e
analisou-se o número de comparações efetuadas envolvendo o valor de cinzento (i.e., a
intensidade) dos pixels das imagens. 
Com esta sequência de testes, foi feita uma tabela, onde é observado o tamanho das imagens que analisamos, 
com os nomes destas ao lado, 
as posições x e y, quando usadas pela função ImagePaste(), o número de comparações, o tempo de execução da função, e 
se deu para localizar a imagem menor na imagem maior, como se pode ver na figura 1:



\begin{figure}[h]
    \centering
    \includegraphics[width=1\textwidth]{imageLocate.png}
    \caption{Testes para a ImageLocateSubImage()}
    \label{fig:Figura 1}
\end{figure}

\subsubsection{Análise formal}


\textbf{Melhor caso:}
O melhor caso, para a função ImageLocateSubImage(), seria se a sub-imagem fosse
encontrada na primeira posição da imagem principal. Nesse caso, a complexidade de tempo
seria a complexidade de tempo da função ImageMatchSubImage, porque só precisaria ser chamada uma vez.
Ou seja, podemos calcular a complexidade do melhor caso, a partir do algoritmo

\begin{align*}
\sum_{\text{srcY}=0}^{h-1} (\sum_{\text{srcX}=0}^{w-1}1)
\end{align*}

\begin{center}
Legenda:

(h = img2->height; w = img2->width)(ImageMatchSubImage())
\end{center}

Dando assim a complexidade de O(hw)=O(n), onde n é o número de pixeis que percorre para a função ImageMatchSubImage().\\






\textbf{Pior caso}\\

O pior caso seria se a sub-imagem não fosse encontrada de todo, ou se fosse encontrada na última posição da imagem principal.
Nesse caso, o algoritmo geral(quando passa pelas duas funções) será:

\begin{align*}
    &\sum_{\text{srcY}=0}^{a-1}[\left \sum_{\text{srcX}=0}^{l-1} (\right \sum_{x=0}^{w - l} (\sum_{y=0}^{h - a}1))]
\end{align*}
\begin{center}

    Legenda:
    (h = img1->height; w = img1->width(ImageMatchSubImage();\\ l =img2->width; a = img1->height)
    \end{center}
\\O algoritmo geral seria O(al)*O((w-l+1)*(h-a+1)), mas considerando n o número de pixeis que percorre para a função ImageMatchSubImage() 
e para a ImageLocateSubImage(), o pior caso seria O(n)*O(n), dando assim uma complexidade de O(n²) (exponencial).\\











\subsubsection{Comparação dos dados experimentais com análise formal}

Para comparar os dados experimentais com a análise formal, um exemplo para o melhor caso,é quando foi feito o teste para localizar uma imagem pequena numa média, onde foi executada a função
ImagePaste(medium,0,0,small), ou seja irá "copiar" a imagem pequena nas coordenadas dos pixels (0,0) na imagem média, como 
se pode observar na segunda linha da Figura ~\ref{fig:Figura 1}. Este caso, é considerado o melhor caso, dado que a partir da fórmula da complexidade(n*m),
podemos confirmar a partir do número de comparações que será igual a n*m(altura da imagem menor* largura da imagem menor) (222*217=48174).

O pior caso possível seria colar a imagem "art3\_222x217.pgm" (222x217) na imagem "airfield-05\_1600x1200.pgm" (1600x1200)
na posição mais à direita e mais abaixo possível. Isso forçaria a função ImageLocateSubImage() a percorrer
quase toda a imagem "large" antes de encontrar a imagem "small".



\subsection{ImageBlur()}

\subsubsection{Dados experimentais}



Para analisar os dados experimentais da primeira e da segunda versão da ImageBlur(), foi realizado uma sequência de testes, com imagens e filtros de diferentes tamanhos, de seguida foi
analisada a evolução do número de operações que consideramos relevantes, neste caso, o número de multiplicações e o número de somas. Para além 
disto, foi registado o tempo de execução da função, e as dimensões do filtro aplicado para cada teste.

\newpage

\subsubsubsection{\textbf{Primeira versão da ImageBlur}}

\begin{figure}[h]
    \centering
 \includegraphics[width=1.3\textwidth]{blur.png}
    \caption{Testes para a ImageBlur()}
    \label{fig:Figura 2}
\end{figure}





\subsubsubsection{\textbf{Segunda versão da ImageBlur}}
\begin{figure}[h]
    \centering
    \includegraphics[width=1.3\textwidth]{blur2.png}
    \caption{Testes para a ImageBlurv2()}
    \label{fig:Figura 3}
\end{figure}

\subsubsection{Análise formal}

\subsubsubsection{\textbf{Primeira versão da ImageBlur}}


Podemos calcular a complexidade da primeira versão da ImageBlur(), a partir do seguinte algoritmo:


\begin{align*}
    &\sum_{\text{y}=0}^{h}[\left \sum_{\text{x}=0}^{w} (\right \sum_{j=-dy}^{dy} (\sum_{i=-dx}^{dx}1))]
\end{align*}

\begin{center}
    Legenda: 
    (h e w são a altura e largura da imagem, dx e dy são as dimensões do filtro aplicado)
\end{center}

\textbf{Melhor caso}\\
O melhor caso para a função ImageBlur() ocorre quando 2dy+1 e 2dx +1 são 1, ou seja a complexidade para 
o melhor caso seria: O((h*w)*(2dy+1)*(2dx+1)) = O(h*w) = O(n), sendo n o número de pixeis. Logo, 
o gráfico tempo(eixo 0y)-(h*w)(eixo 0x) será linear.

 \textbf{Pior caso}

O pior caso para a função ImageBlur(), dado que dentro desta é chamada a função ImageValidPos(), então irá 
verificar se 2dy+1 e 2dx+1 são iguais ou menores que a altura e largura, respetivamente. Portanto, 
a complexidade do pior caso seria O((h*w)*(h)*(w)) = O((h*w)²)=O(n²), sendo n o número de pixeis que percorre a função ImageBlur(), ou seja 
o gráfico tempo(eixo 0y)-(h*w)(eixo 0x) será exponencial.\\









\subsubsubsection{\textbf{Segunda versão da ImageBlur}}\\

Podemos calcular a complexidade da segunda versão da ImageBlurv2(), a partir do seguinte algoritmo:

\begin{align*}
    \sum_{\text{y}=0}^{h} (\sum_{\text{x}=0}^{w}1) + \sum_{\text{y}=0}^{h} (\sum_{\text{x}=0}^{w}1) 
    \end{align*}
    
    \begin{center}
    Legenda:
    
    (h e w são a altura e largura da imagem)
    \end{center}
    
    Dando assim a complexidade de O(hw+hw)=O(2hw)=O(2n)=O(n), onde n é o número de pixeis que percorre para a função ImageBlurv2().
    Por isso, o gráfico tempo(eixo 0y)-(h*w)(eixo 0x) será linear.


\subsubsection{Comparação dos dados experimentais com análise formal}

A função ImageBlur() tem um tempo de execução maior devido à sua abordagem menos eficiente, realizando um grande número de operações de soma ao calcular a média dos pixels vizinhos para cada pixel da imagem.

Por outro lado, ImageBlurv2(), embora realize mais operações de multiplicação devido ao uso de uma imagem integral para cálculos de média, tem um tempo de execução significativamente menor. Isso se deve à sua abordagem mais eficiente, que reduz o número total de operações necessárias.

Em resumo, ImageBlurv2() é mais eficiente que ImageBlur(), apesar de realizar mais operações de multiplicação.\\

\subsubsubsection{\textbf{Análise Comparativa Algoritmo Básico/Algoritmo Melhorado}}

Comparando o algoritmo básico da função ImageBlur() com o algoritmo melhorado da função ImageBlurv2(),
fazendo os mesmos testes com as mesmas imagens e dimensões, como vemos na Figura~\ref{fig:Figura 2} e na Figura ~\ref{fig:Figura 3}, 
o tempo de execução da imageBlur() será muito maior que a imageBlurv2(), dado que, como o algoritmo não é tão eficiente,
será mais lento. Para além disto,a ImageBlur() realiza mais somas(variável global criada no image8bit.h) porque para cada pixel da imagem, 
esta soma os valores dos pixeis vizinhos. Isto é feito para todos os pixeis da imagem, resultando
num grande número de operações de soma. Contudo, a ImageBlurv2() realiza mais multiplicações(outra variável global criada no image8bit.h)
do que a ImageBlur(), visto que, a segunda versão da função utiliza uma imagem integral para calcular
a média dos pixels vizinhos de forma mais eficiente. No entanto, o cálculo da imagem integral envolve
mais operações de multiplicação. Por exemplo, para calcular o número de pixels na janela de desfoque,
a função multiplica a largura pela altura da janela. Além disso, para calcular a média dos pixels na
janela de desfoque, a função divide a soma dos pixels pelo número de pixels, que é uma operação de 
multiplicação (já que a divisão é equivalente a multiplicar pelo inverso).\\








\section{Conclusão}
Este documento aborda o TAD "image8bit," concentrando-se na implementação de funções para a manipulação eficiente de imagens em 
tons de cinza. A função ImageLocateSubImage() foi analisada em termos de complexidade, evidenciando casos melhores e piores. Quanto às 
funções de desfoque, ImageBlur() e ImageBlurv2(), os testes comparativos revelaram que a segunda versão é mais eficiente em tempo de execução,
 embora envolva mais operações de multiplicação. Em suma, este trabalho destaca a importância da escolha de estratégias para otimizar operações
  específicas no processamento de imagens.

\end{document}y