\documentclass{article}
\usepackage[T1]{fontenc} % Fontes T1
\usepackage[utf8]{inputenc} % Input UTF8
\usepackage[backend=biber, style=ieee]{biblatex} % para usar bibliografia
\usepackage{csquotes}
\usepackage[portuguese]{babel} %Usar língua portuguesa
\usepackage{blindtext} % Gerar texto automaticamente
\usepackage[printonlyused]{acronym}
\usepackage{hyperref} % para autoref
\usepackage{indentfirst}
\usepackage{fancyhdr}
\usepackage{hyperref}


\usepackage{graphicx} % Required for inserting images
\usepackage{float} % For precise figure placement using [H]

\title{TAD Image8bit - Algoritmo e Estruturas de Dados}
\author{
  Joaquim Martins: 115931 \\
  David Pelicano: 113391 \\
  Tiago Soure: 108769
}
\date{November 2023}

\usepackage{amsmath}
\begin{document}

\maketitle

\begin{figure}[H]
  \centering
  \includegraphics[scale=0.5]{image.png}
\end{figure}  

\newpage
\renewcommand\contentsname{Índice}
\tableofcontents 
\newpage

\pagenumbering{arabic}

\section{Aspetos Gerais}

\section{Apresentação}
\subsection{ImageLocateSubImage()}

A função ImageLocateSubImage(Image img1, int* px, int* py, Image img2) procura uma sub-imagem(img2) dentro
de uma imagem maior(img1). Esta percorre todas as posições possíveis na imagem img1 onde a subimagem img2
poderia se encaixar. 
Para cada posição, ela chama a função ImageMatchSubImage() para verificar se a subimagem img2 
corresponde à parte da imagem img1 na posição atual.
Se uma correspondência é encontrada, a função define os valores dos inteiros apontados por px e
py para as coordenadas x e y da posição correspondente na imagem img1 e retorna 1.
Se a função percorre todas as posições possíveis na imagem img1 sem encontrar uma correspondência,
ela retorna 0, indicando que a subimagem img2 não foi encontrada na imagem img1. Nesse caso, os valores
dos inteiros apontados por px e py não são alterados.
Nos testes, para retornar 1, foi usada a função ImagePaste(Image img1, int x, int y, Image img2),
para colar uma imagem(img2) numa posição específica(x,y) dentro de uma imagem maior(img1), copiando
cada pixel de img2 para a posição correspondente na img1.

\subsection{ImageBlur()}

A função ImageBlur() será usada para aplicar um efeito de desfoque numa imagem.
Esta percorre cada pixel da imagem e calcula a média dos pixeis numa janela à volta do pixel atual.
Contudo, foram feitas dois algoritmos, um mais básico e outro melhorado, ou seja foram criadas 2 funções:
ImageBlur() e ImageBlurv2(). A principal diferença entre as duas funções é a maneira como estas calculam essa média.

Para o algoritmo mais básico, ou seja, para a função ImageBlur(), é usado um método direto para calcular a média.
A função percorre cada pixel na área ao redor do píxel atual, soma os seus valores e divide pelo número total de pixeis na área. 
Este processo é realizado para cada píxel na imagem. A função utiliza uma imagem temporária para
 armazenar os novos valores dos pixeis, de modo a não modificar os valores dos píxeis que ainda 
 precisam ser processados.


Para a função ImageBlurv2() foi utilizada uma técnica denominada imagem integral para calcular a média de 
 forma mais eficiente. Uma imagem integral é uma matriz na qual o valor em cada ponto é a soma de 
 todos os pixeis acima e à esquerda do ponto atual, incluindo o próprio ponto. Isto possibilita o 
 cálculo da soma de qualquer retângulo na imagem em tempo constante, apenas subtraindo e adicionando 
 os valores apropriados da imagem integral. Este método torna o cálculo da média significativamente mais
  rápido, especialmente para imagens grandes.














\section{Análise da complexidade}
\subsection{ImageLocateSubImage()}





Após concluir o desenvolvimento do TAD, é necessário analisar a eficiência computacional do
algoritmo desenvolvido para a função ImageLocateSubImage() e da função ImageBlur()/ImageBlurv2().
\subsubsection{Dados experimentais}
Para obter os dados experimentais, realizou-se uma sequência de testes, com imagens de diferentes tamanhos, onde registou-se e
analisou-se o número de comparações efetuadas envolvendo o valor de cinzento (i.e., a
intensidade) dos pixels das imagens. 
Com esta sequência de testes, foi feita uma tabela, onde é observado o tamanho das imagens que analisamos, 
com os nomes destas ao lado, 
as posições x e y, quando usadas pela função ImagePaste(), o número de comparações, o tempo de execução da função, e 
se deu para localizar a imagem menor na imagem maior, como se pode ver na figura 1:



\begin{figure}[h]
    \centering
    \includegraphics[width=1.3\textwidth]{imageLocate.png}
    \caption{Testes para a ImageLocateSubImage()}
    \label{fig:Figura 1}
\end{figure}


\subsubsection{Análise formal}

\textbf{Melhor caso:}
O melhor caso, para a função ImageLocateSubImage(), seria se a sub-imagem fosse
encontrada na primeira posição da imagem principal. Nesse caso, a complexidade de tempo
seria a complexidade de tempo da função ImageMatchSubImage, porque só precisaria ser chamada uma vez.
Ou seja, podemos calcular a complexidade do melhor caso, a partir do algoritmo

\begin{align*}
\sum_{\text{srcY}=0}^{h-1} (\sum_{\text{srcX}=0}^{w-1}1)
\end{align*}

\begin{center}
Legenda:

(h-1 = img2->height-1; w-1 = img2->width-1)(ImageMatchSubImage())
\end{center}

Dando assim a complexidade de O(nm), onde n é a altura da imagem2(menor)
e m é a largura da imagem2(menor).


Um exemplo para o melhor caso,é quando foi feito o teste para localizar uma imagem pequena numa média, onde foi executada a função
ImagePaste(medium,0,0,small), ou seja irá "copiar" a imagem pequena nas coordenadas dos pixels (0,0) na imagem média, como 
se pode observar na segunda linha da figura1. Este caso, é considerado o melhor caso, dado que a partir da fórmula da complexidade(n*m),
podemos confirmar a partir do número de comparações que será igual a n*m(altura da imagem menor* largura da imagem menor).



\textbf{Pior caso:}

O pior caso seria se a sub-imagem não fosse encontrada de todo, ou se fosse encontrada na última posição da imagem principal.
Nesse caso, o algoritmo será:

\begin{align\times}
    &\sum_{\text{srcY}=0}^{h-1}[\left \sum_{\text{srcX}=0}^{w-1} \right \sum_{x=0}^{w - n + 1} (\sum_{y=0}^{h - m + 1}1)]
\end{align\times}

    
    \begin{center}
    Legenda:
    
    (h-1 = img2->height-1; w-1 = img2->width-1(ImageMatchSubImage()); w-n+1 =img1->width - img2->width + 1("0 a <=")); w-n+1 =img1->height - img2->height + 1("0 a <=")
    \end{center}

Dando assim a complexidade de O((al)\times(w-l+1)\times(h-a+1)), onde a é a altura da imagem2(menor)
e l é a largura da imagem2(menor), h é a altura da imagem1(maior) e w é a largura da imagem1(maior). 






\subsubsection{Comparação dos dados experimentais com análise formal}

\subsection{ImageBlur()}

\subsubsection{Dados experimentais}

\subsubsubsection{Primeira versão da ImageBlur}

\begin{figure}[h]
    \centering
    \includegraphics[width=1.3\textwidth]{blur.png}
    \caption{Testes para a ImageBlur()}
    \label{fig:Figura 2}
\end{figure}





\subsubsubsection{Segunda versão da ImageBlur}
\begin{figure}[h]
    \centering
    \includegraphics[width=1.3\textwidth]{blur2.png}
    \caption{Testes para a ImageBlurv2()}
    \label{fig:Figura 3}
\end{figure}
\subsubsection{Análise formal}
\subsubsubsection{Primeira versão da ImageBlur}

Podemos calcular a complexidade do melhor caso, a partir do seguinte algoritmo:


\begin{align*}
    &\sum_{\text{y}=0}^{h}[\left \sum_{\text{x}=0}^{w} \right \sum_{j=-dy}^{dy} (\sum_{i=-dx}^{dx}1)]
\end{align*}
 
\begin{center}
    Legenda:
    
    (h e w são a altura e largura da imagem, dx e dy são as dimensões do filtro aplicado)
    \end{center}


\textbf{Melhor caso:}
O melhor caso para a função ImageBlur() ocorre quando 2dy+1 e 2dx +1 são 1, ou seja a complexidade para 
o melhor caso seria: O((h*w)*(2dy+1)*(2dx+1)) = O(h*w) = O(n), sendo n o número de pixeis. Logo, 
o gráfico tempo(eixo 0y)-(h*w)(eixo 0x) será exponencial.

 \textbf{Pior caso}

O pior caso para a função ImageBlur(), dado que dentro desta é chamada a função ImageValidPos(), então irá 
verificar se 2dy+1 e 2dx+1 são iguais ou menores que a altura e largura, respetivamente. Portanto, 
a complexidade do pior caso seria O((h*w)*(h)*(w)) = O((h*w)²)=O(n²), sendo n o número de pixeis, ou seja 
o gráfico tempo(eixo 0y)-(h*w)(eixo 0x) será exponencial.

\subsubsubsection{Segunda versão da ImageBlur}

Podemos calcular a complexidade do melhor caso, a partir do seguinte algoritmo:

\begin{align*}
    \sum_{\text{y}=0}^{h} (\sum_{\text{x}=w}^{w}1) + \sum_{\text{y}=0}^{h} (\sum_{\text{x}=w}^{w}1) 
    \end{align*}
    
    \begin{center}
    Legenda:
    
    (h e w são a altura e largura da imagem)
    \end{center}
    
    Dando assim a complexidade de O(hw+hw)=O(2hw)=O(2n)=O(n), onde n é o número de pixels.
    Por isso, o gráfico tempo(eixo 0y)-(h*w)(eixo 0x) será linear.

\subsubsection{Análise Comparativa Algoritmo Básico/Algoritmo Melhorado}

Comparando o algoritmo básico da função ImageBlur() com o algoritmo melhorado da função ImageBlurv2(),
fazendo os mesmos testes com as mesmas imagens e dimensões, como vemos na Figura~\ref{fig:Figura 2} e na Figura ~\ref{fig:Figura 3}, 
o tempo de execução da imageBlur() será muito maior que a imageBlurv2(), dado que, como o algoritmo não é tão eficiente,
será mais lento. Para além disto,a ImageBlur() realiza mais somas(variável global criada no image8bit.h) porque para cada pixel da imagem, 
esta soma os valores dos pixeis vizinhos. Isto é feito para todos os pixeis da imagem, resultando
num grande número de operações de soma. Contudo, a ImageBlurv2() realiza mais multiplicações(outra variável global criada no image8bit.h)
do que a ImageBlur(), visto que, a segunda versão da função utiliza uma imagem integral para calcular
a média dos pixels vizinhos de forma mais eficiente. No entanto, o cálculo da imagem integral envolve
mais operações de multiplicação. Por exemplo, para calcular o número de pixels na janela de desfoque,
a função multiplica a largura pela altura da janela. Além disso, para calcular a média dos pixels na
janela de desfoque, a função divide a soma dos pixels pelo número de pixels, que é uma operação de 
multiplicação (já que a divisão é equivalente a multiplicar pelo inverso).

































\section{Conclusão}

O desenvolvimento do TAD Image8bit foi concluído com sucesso, proporcionando funcionalidades robustas para manipulação de imagens de 8 bits. A análise de complexidade revelou eficiência nas operações implementadas. Possíveis melhorias e otimizações foram identificadas para futuras iterações.

\section{Bibliografia}

Inclui referências bibliográficas relevantes consultadas durante o desenvolvimento do TAD Image8bit.

\end{document}