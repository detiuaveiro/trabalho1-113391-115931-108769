\documentclass{article}
\usepackage{graphicx} % Required for inserting images
\usepackage{float} % For precise figure placement using [H]

\title{TAD Image8bit - Algoritmo e Estruturas de Dados}
\author{
  Joaquim Martins: 115931 \\
  David Pelicano: 113391 \\
  Tiago Soure: 108769
}
\date{November 2023}

\begin{document}

\maketitle

\begin{figure}[H]
  \centering
  \includegraphics[scale=0.5]{image.png}
\end{figure}  

\newpage
\renewcommand\contentsname{Índice}
\tableofcontents 
\newpage

\pagenumbering{arabic}

\section{Desenvolvimento do TAD Image8bit}

\subsection{Análise dos Arquivos Fornecidos}
Os arquivos \texttt{image8bit.h} e \texttt{image8bit.c} foram analisados para compreensão da estrutura e das funções fornecidas.

\subsection{Implementação das Funções}
As funções especificadas no arquivo de interface \texttt{image8bit.h} foram completadas no arquivo de implementação \texttt{image8bit.c}. Funções auxiliares foram adicionadas conforme necessário para garantir a correta execução.

\subsection{Testes e Resultados}
Foram realizados testes abrangentes em todas as funções desenvolvidas, utilizando os programas de teste fornecidos. Os resultados foram analisados para verificar a corretude e eficiência do TAD Image8bit.

\section{Análise da Complexidade da Função \texttt{ImageLocateSubImage()}}

\subsection{Testes de Desempenho}
Para obter dados experimentais sobre a complexidade da função \texttt{ImageLocateSubImage()}, tivemos de criar um novo ficheiro C (\texttt{testes.c}) onde executamos a função com várias imagens de diferentes tamanhos e medir o tempo de execução para cada operação. Isto foi feito com a ajuda da função \texttt{cpu\_time()} como se pode ver na Figura 1:









\subsection{Análise Formal de Complexidade}
A complexidade do algoritmo foi analisada para o melhor e o pior caso, considerando todas as possíveis posições na imagem principal.

\subsection{Comparação de Resultados}
Os resultados dos testes foram comparados com a análise formal de complexidade para avaliar a eficiência do algoritmo implementado.

\section{Análise da Complexidade da Função \texttt{ImageBlur()}}

\subsection{Testes de Desempenho}
Testes foram conduzidos utilizando imagens e filtros de diferentes tamanhos. O número de operações relevantes foi registrado para analisar a eficiência do algoritmo.

\subsection{Análise Formal de Complexidade}
A complexidade do algoritmo foi analisada, considerando o tamanho da janela usada para desfocar a imagem.

\subsection{Comparação de Estratégias}
Apesar de não ser a implementação mais imediata, foram comparadas diferentes estratégias algorítmicas para a resolução do problema, destacando as vantagens e desvantagens de cada abordagem.

\section{Resultados dos Testes}

\begin{itemize}
    \item \textbf{ImageBlur: 1ª função (tempo de execução maior/mais lenta)}
    \begin{itemize}
        \item Tempo de execução (imagem pequena): 0.204621 segundos
        \item Tempo de execução (imagem média): 1.229907 segundos
        \item Tempo de execução (imagem larga): 8.457981 segundos
    \end{itemize}
    
    \item \textbf{ImageBlur: 1ª função (tempo de execução maior/mais lenta) com dimensões máximas da imagem}
    \begin{itemize}
        \item Tempo de execução (imagem pequena): 68.191713 segundos
        \item Os tempos de execução das imagens médias e largas não foram incluídos devido à ineficiência da primeira versão da função \texttt{ImageBlur}, especialmente quando aplicada às dimensões máximas da imagem, resultando em longos períodos de espera.
    \end{itemize}
    
    \item \textbf{ImageBlurv2: 2ª função (tempo de execução menor/mais rápida)}
    \begin{itemize}
        \item Tempo de execução (imagem pequena): 0.004255 segundos
        \item Tempo de execução (imagem média): 0.025867 segundos
        \item Tempo de execução (imagem larga): 0.126103 segundos
    \end{itemize}
    
    \item \textbf{ImageBlurv2: 2ª função (tempo de execução menor/mais rápida) com dimensões máximas da imagem}
    \begin{itemize}
        \item Tempo de execução (imagem pequena): 0.002837 segundos
        \item Tempo de execução (imagem média): 0.021988 segundos
        \item Tempo de execução (imagem grande): 0.216548 segundos
    \end{itemize}
\end{itemize}

\section{Conclusão}

O desenvolvimento do TAD Image8bit foi concluído com sucesso, proporcionando funcionalidades robustas para manipulação de imagens de 8 bits. A análise de complexidade revelou eficiência nas operações implementadas. Possíveis melhorias e otimizações foram identificadas para futuras iterações.

\section{Bibliografia}

Inclui referências bibliográficas relevantes consultadas durante o desenvolvimento do TAD Image8bit.

\end{document}
